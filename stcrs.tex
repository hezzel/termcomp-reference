\documentclass{lmcs}
\pdfoutput=1

\usepackage{enumerate}
\usepackage[colorlinks=true]{hyperref}
\usepackage{amssymb}
\usepackage{xcolor,latexsym,amsmath,extarrows,alltt}
\usepackage{xspace}
\usepackage{booktabs}
\usepackage{mathtools}
\usepackage{enumitem}
\usepackage{stmaryrd}
\usepackage{microtype}

\theoremstyle{theorem}\newtheorem{theorem}{Theorem}
\theoremstyle{theorem}\newtheorem{lemma}[theorem]{Lemma}
\theoremstyle{theorem}\newtheorem{corollary}[theorem]{Corollary}
\theoremstyle{definition}\newtheorem{definition}[theorem]{Definition}
\theoremstyle{definition}\newtheorem{example}[theorem]{Example}

\newcommand{\N}{\mathbb{N}}
\newcommand{\F}{\mathcal{F}}
\newcommand{\V}{\mathcal{V}}
\newcommand{\Vmeta}{\mathcal{M}}
\newcommand{\Vfree}{\mathcal{V}_{\mathit{nonb}}}
\newcommand{\Vbound}{\mathcal{V}}
\newcommand{\Sorts}{\mathcal{S}}
\newcommand{\Types}{\mathcal{T}\!ypes}
\newcommand{\Terms}{\mathcal{T}}
\newcommand{\ATerms}{\mathcal{T}_{\mathcal{A}}}
\newcommand{\FOTerms}{\mathcal{T}_{\mathcal{FO}}}
\newcommand{\Rules}{\mathcal{R}}
\newcommand{\FV}{\mathit{FV}}
\newcommand{\FMV}{\mathit{MVar}}
\newcommand{\Positions}{\mathit{Positions}}
\newcommand{\SubPositions}{\mathit{SubPositions}}
\newcommand{\HeadPositions}{\mathit{HeadPositions}}
\newcommand{\Pairs}{\mathit{Pairs}}

\newcommand{\domain}{\mathtt{domain}}
\newcommand{\order}{\mathit{order}}

\newcommand{\asort}{\iota}
\newcommand{\bsort}{\kappa}
\newcommand{\atype}{\sigma}
\newcommand{\btype}{\tau}
\newcommand{\ctype}{\pi}
\newcommand{\dtype}{\alpha}
\newcommand{\identifier}[1]{\mathtt{#1}}
\newcommand{\afun}{\identifier{f}}
\newcommand{\bfun}{\identifier{g}}
\newcommand{\cfun}{\identifier{h}}
\newcommand{\avar}{x}
\newcommand{\bvar}{y}
\newcommand{\cvar}{z}
\newcommand{\Avar}{X}
\newcommand{\Bvar}{Y}
\newcommand{\Cvar}{Z}
\newcommand{\AFvar}{F}
\newcommand{\BFvar}{G}
\newcommand{\CFvar}{H}
\newcommand{\typeof}{\mathsf{typeof}}

\newcommand{\clause}[1]{\textbf{#1}}
\newcommand{\hastype}{\ \mathtt{::}\ }

\newcommand{\abs}[3]{\lambda #1\!:\!#2.#3}
\newcommand{\app}[2]{#1 \cdot #2}
\newcommand{\apps}[3]{#1 \cdot #2 \cdots #3}
\newcommand{\metaabs}[2]{\langle #1\rangle#2}
\newcommand{\metaabsn}[4]{\metaabs{#1_1\!:\!#2_1,\dots,#1_n\!:\!#2_{#3}}{#4}}
\newcommand{\metaabslist}[3]{\langle #1_1\!:\!#2_1,\dots,#1_n\!:\!#2_{#3} \rangle}

\newcommand{\arity}{\mathit{ar}}
\newcommand{\head}{\mathsf{head}}
\newcommand{\arrtype}{\rightarrow}
\newcommand{\arrz}{\Rightarrow}
\newcommand{\arr}[1]{\arrz_{#1}}
\newcommand{\arrr}[1]{\arr{#1}^*}
\newcommand{\subtermeq}{\unlhd}
\newcommand{\headsubtermeq}{\unlhd_{\bullet}}
\newcommand{\supterm}{\rhd}
\newcommand{\suptermeq}{\unrhd}
\newcommand{\replace}[3]{#1_{#2:=#3}}

\newcommand{\symb}[1]{\mathtt{#1}}

\newcommand{\nul}{\symb{0}}
\newcommand{\one}{\symb{1}}
\newcommand{\nil}{\symb{nil}}
\newcommand{\cons}{\symb{cons}}
\newcommand{\strue}{\symb{true}}
\newcommand{\sfalse}{\symb{false}}
\newcommand{\append}{\symb{append}}
\newcommand{\suc}{\symb{s}}
\newcommand{\map}{\symb{map}}
\newcommand{\bool}{\symb{bool}}
\newcommand{\nat}{\symb{nat}}
\newcommand{\real}{\symb{real}}
\newcommand{\lijst}{\symb{list}}
\newcommand{\unitsort}{\mathtt{o}}

\newcommand{\cora}{\textsf{CORA}\xspace}

\newcommand{\secshort}{\S}
\newcommand{\myparagraph}[1]{\paragraph{\textbf{#1}}}

\setlength{\parindent}{0pt}
\setlength{\parskip}{\bigskipamount}
\setlist[itemize]{topsep=-\bigskipamount}

\newcommand{\mysubsection}[1]{\vspace{-12pt}\subsubsection{#1}}

\begin{document}

\title{Higher-order rewriting (union beta)}

\maketitle

\begin{abstract}
The following pages describe the syntax and semantics of the higher-order ARI formalism.
This format is simply-typed (without formalism) and in principle curried.  Matching is
\emph{mostly} plain, though a limited form of higher-order matching modulo beta is
included for pattrens.
\end{abstract}

\section{Syntax}

\subsection{Grammar}
\ 

\begin{tabular}{lll}
trs & ::= & \texttt{(format higher-order)} sort+ fun+ hrule+ \\
sort & ::= & \texttt{(sort} IDENTIFIER\texttt{)} \\
fun & ::= & \texttt{(fun} IDENTIFIER type\texttt{)} \\
type & ::= & IDENTIFIER $|$ \texttt{(-}$\mathtt{>}$ type+ IDENTIFIER\texttt{)} \\
term & ::= & IDENTIFIER $|$ \texttt{(}IDENTIFIER term+\texttt{)} $|$ \texttt{(}\texttt{lambda} \texttt{(}var+\texttt{)} term\texttt{)} \\
var & ::= & \texttt{(}IDENTIFIER type\texttt{)} \\
rule & ::= & \texttt{(rule} term term\texttt{)} \\
\end{tabular}

\subsection{Restrictions}

\begin{itemize}
\item The left- and right-hand sides of rules must have the same type.
\item All unbound variables in the right-hand side must occur in the left-hand side.
\item The left-hand side has the form \texttt{f} or (\texttt{f} l1 ... ln) with f a function symbol.
\item The left-hand side must be a \textbf{pattern}, i.e., the arguments of every applied free variable are distinct bound variables.
\item If any variable occurs applied on the left, then each occurrence in the left-hand side must have exactly the same number of
  arguments, and each occurrence on the right must have \emph{at least} that number of arguments.
\end{itemize}

\section{Semantics}

\subsection{Types}

We use simple types, i.e., fixing a set $\Sorts$ of sorts, the types $\Types$ of \emph{types} is generated inductively by:

\[ \Types ::= \Sorts\, |\, \Types \arrtype \Types \]

The arrow operator $\arrtype$ is right-associative, so all types can be denoted in a form
$\atype_1 \arrtype \dots \arrtype \atype_m \arrtype \asort$ with $\asort \in \Sorts$.
In ARI format, such a type is denoted $\mathtt{(-\!\!>\ \sigma_1\ \dots\ \sigma_n\ \asort)}$.

\subsection{Terms}

We fix a set $\F$ of \emph{function symbols}, also called the \emph{alphabet}; each function
symbol is equipped with a type.  Notation: $(\afun \hastype \atype) \in \F$, where $\afun$ is
the name and $\atype$ the type.  We also fix an infinite set $\V$ of \emph{variables}, disjoint
from the names used in $\F$.

A \emph{type environment} is a function $\Gamma$ from a subset of $\V$ to $\Types$.
(That is, it is a set of pairs $(\avar \hastype \atype)$ such that no $\avar$ occurs more than
once.)

\emph{Terms} are those expressions $s$ such that $\Gamma \vdash s \hastype \atype$ can be derived
for some type environment $\Gamma$ using the following clauses:

\begin{description}
\item[constant] if $(\afun \hastype \atype) \in \F$ then $\Gamma \vdash \afun \hastype \atype$
\item[variable] if $(\avar \hastype \atype) \in \Gamma$ then $\Gamma \vdash \avar \hastype \atype$
\item[application] if $\Gamma \vdash s \hastype \atype \arrtype \btype$ and
  $\Gamma \vdash t \hastype \atype$ then $\Gamma \vdash \app{s}{t} :: \btype$
\item[abstraction] if $\avar \in \V$ and $\replace{\Gamma}{\avar}{\atype} \vdash t :: \btype$, then
  $\Gamma \vdash \abs{\avar}{\atype}{t} :: \atype \arrtype \btype$ \\
  (Here, $\replace{\Gamma}{\avar}{\atype}$ is the function that maps $\avar$ to $\atype$ and all other
  variables $\bvar$ to $\Gamma(\bvar)$.)
\end{description}

The application operator $\cdot$ is left-associative, so every term can be written $h \cdot s_1
\cdots s_n$ with $n \geq 0$, where $h$ is a function symbol, variable or abstraction.
In ARI format, such a term is denoted $(h\ s_1 \cdots s_n)$.
%The abstraction operator extends as far to the right as possible, so a term
%$\abs{x}{\atype}{s \cdot t \cdot q}$ should be read as $\abs{x}{\atype}{((s \cdot t) \cdot q)}$.
We also use $\lambda x_1\!:\!\sigma_1,\dots,x_n\!:\!\sigma_n.s$ to denote
$\abs{x_1}{\sigma_1}{(\lambda \dots (\abs{x_n}{\sigma_n}{s})\dots)}$.  In ARI format, this is denoted
$\mathtt{(lambda\ ((}x_1\ \sigma_1\mathtt{)} \dots \mathtt{(}x_n\ \sigma_n\mathtt{))}\ s\mathtt{)}$.

Term equality is considered modulo $\alpha$-equivalence as usual (where variable renaming does not
affect types, so for instance $\abs{\avar}{\atype}{\symb{0}} \neq \abs{\bvar}{\btype}{\symb{0}}$ if
$\atype \neq \btype$).  Equality is \emph{not} considered modulo $\beta$ or $\eta$-equivalence, so
for instance $\app{(\abs{\avar}{\nat}{(\app{\symb{s}}{\avar})})}{\symb{0}}$ is distinct from
$\app{\symb{s}}{\symb{0}}$ (but \emph{is} equal to
$\app{(\abs{\bvar}{\nat}{(\app{\symb{s}}{\bvar})})}{\symb{0}}$).

We define the set $\FV(s)$ of free variables in $s$ as usual: the set of all variables occurring in
$s$ which are not bound by a $\lambda$.

When reasoning about terms, it is often useful to consider the ``maximally applied subterm''
relation $\suptermeq$, defined as follows:
\begin{itemize}
\item $s \suptermeq t$ if $s = t$ or $s \supterm t$
\item $\app{s}{u} \supterm t$ if $u \suptermeq t$ or $s \supterm t$
\item $\abs{\avar}{\atype}{s} \supterm t$ if $u \suptermeq t$
\end{itemize}
Note that for instance $\app{\app{\afun}{s_1}}{s_2} \supterm s_i$ but \emph{not} $\app{\app{\afun}{
s_1}}{s_2} \supterm \app{\afun}{s_1}$: subterms at the head of an application are deliberately not
included.

\subsection{Substitution and variable capture}

For type environments $\Gamma,\Delta$, a \emph{substitution} is a type-preserving function from
$\Gamma$ to the set of terms typable under $\Delta$.  Formally, we denote $\Gamma, \Delta \vdash
\gamma$ if $\domain(\gamma) = \domain(\Gamma)$ and $\Delta \vdash \gamma(\avar) \hastype \atype$
for every $(\avar \hastype \atype) \in \Gamma$.

A substitution $\gamma$ is applied to a term $s$, notation $s\gamma$, by the following procedure:
\begin{enumerate}
\item rename all bound variables in $s$ to fresh variables, i.e., distinct variables that do not
  occur in $\Gamma$ or $\Delta$
\item replace each free variable $x$ by $\gamma(x)$
\end{enumerate}

Formally, if $\Gamma,\Delta \vdash \gamma$ and $\Gamma \vdash s \hastype \atype$ and all bound
variables in $s$ have been renamed to be disjoint from each other and from the variables in
$\Gamma$ and $\Delta$, then $s\gamma$ is obtained by the following clauses, yielding a term such
that $\Delta \vdash s\gamma \hastype \atype$:

\begin{itemize}
\item $\afun\gamma = \afun$
\item $\avar\gamma = \gamma(\avar)$ if $\avar \in \domain(\gamma)$
\item $\avar\gamma = \avar$ if $\avar \notin \domain(\gamma)$
\item $(\app{s}{t})\gamma = \app{(s\gamma)}{(t\gamma)}$
%\item $(\abs{\avar}{\atype}{s})\gamma = \abs{\bvar}{\atype}{(s\replace{\gamma}{\avar}{\bvar})}$
%  for $\bvar$ a \emph{fresh} variable, i.e., $\bvar$ does not occur in $\Gamma$ or $\Delta$.
\item $(\abs{\avar}{\atype}{s})\gamma = \abs{\avar}{\atype}{(s\gamma)}$;
  here, due to the $\alpha$-renaming beforehand, $\avar$ does not occur in $\domain(\gamma)$ or in
  $\FV(\bvar\gamma)$ for any $\bvar \in \FV(s) \setminus \{\avar\}$
\end{itemize}

Note that substitution is preserved under $\alpha$-renaming, i.e., if $s =_\alpha t$, and $\delta$
is a substitution on the same domain as $\gamma$ such that each $\gamma(\avar) =_\alpha
\delta(\avar)$, then $s\gamma =_\alpha t\delta$.

We denote $[\avar_1:=s_1,\dots,\avar_n:=s_n]$ for the substitution on domain $\{\avar_1,\dots,
\avar_n\}$ that maps each $\avar_i$ to $s_i$.

\begin{example}
Regarding variable capture, consider the following two illustrative cases.

\begin{itemize}
\item $(\abs{\avar}{\atype}{(\app{\afun}{\avar})})[\avar:=\symb{a}] =
  \abs{\cvar}{\atype}{(\app{\afun}{\cvar})} =_\alpha
  \abs{\avar}{\atype}{(\app{\afun}{\avar})}$: \\
  the bound variable $\avar$ is not replaced by $\symb{a}$, since it is bound
  (and hence renamed before substituting)
\item $(\abs{\avar}{\atype}{(\app{\afun}{\bvar})})[\bvar:=\avar] =
  \abs{\cvar}{\atype}{(\app{\afun}{\avar})}$: \\
  the binder is renamed, so the $\avar$ in the range of the substitution is not captured.
\end{itemize}
\end{example}

\subsection{Rules}

A \emph{rule} $\rho$ consists of two terms $\ell \arrz r$, such that a type environment
$\Gamma$ exists with:

\begin{itemize}
\item both sides have the same type: both $\Gamma \vdash \ell \hastype \atype$ and $\Gamma \vdash
  r \hastype \atype$ for some $\atype$ (but it is not required that $\atype$ is a sort)
\item all variables on the right also occur on the left: $\FV(r) \subseteq \FV(\ell)$
\item the left-hand side is a functional term: $\ell$ has a form $\apps{\afun}{\ell_1}{\ell_k}$
  with $\afun \in \F$
\item $\ell$ is a \emph{pattern}: if $\ell \suptermeq \apps{\avar}{s_1}{s_n}$ for some
  $\avar \in \FV(\ell)$ then $s_1,\dots,s_n$ are distinct bound variables; that is,
  $\{s_1,\dots,s_n\} \subseteq \V \setminus \FV(\ell)$ and $s_i \neq s_j$ for
  $1 \leq i < j \leq n$
\item free variables are used \emph{consistently}: there is a function $\arity$ from $\FV(\ell)$
  to $\N$ such that:
  \begin{itemize}
  \item if $\ell \suptermeq \apps{\avar}{\bvar_1}{\bvar_n}$ with $\avar \in \FV(\ell)$ then
    $\arity(\avar) = n$
  \item if $r \suptermeq \apps{\avar}{s_1}{s_n}$ with $\avar \in \FV(\ell)$ then
    $\arity(\avar) \leq n$
  \end{itemize}
\item both left- and right-hand sides are $\beta$-normal: neither has a subterm of the form
  $\app{(\abs{\avar}{\atype}{s})}{t_0} \cdots t_n$
\end{itemize}

Note that $\Gamma$ is always uniquely defined by the restrictions on $\ell$.

\begin{example}
A valid rule where all variables have arity $0$ is:
\[
\app{\app{\symb{map}}{F}}{(\app{\app{\cons}{x}}{y})} \arrz
\app{\app{\cons}{(\app{F}{x})}}{(\app{\app{\symb{map}}{F}}{y})}
\]
A valid rule where $F$ has arity $1$ is:
\[
\app{\app{\symb{map}}{(\abs{\cvar}{\nat}{(\app{F}{\cvar})})}}{(\app{\app{\cons}{x}}{y})} \arrz
\app{\app{\cons}{(\app{F}{x})}}{(\app{\app{\symb{map}}{(\abs{\cvar}{\nat}{(\app{F}{\cvar})})}}{y})}
\]
Note that these rules model \emph{essentially} the same relation.  To keep the database clean, we
advise submitting only examples using the shorter form where this is possible (that is, keeping
variable arity as low as possible by eta-shortening variables), unless the choice of presentation
is actively relevant for questions of termination or confluence.

An example of an invalid rule is the following:
\[
\app{\symb{der}}{(\abs{\avar}{\real}{(\app{\app{\symb{plus}}{(\app{F}{\avar})}}{(\app{G}{\avar})})})}
\arrz
\abs{\avar}{\real}{\app{\app{\symb{plus}}{(\app{(\app{\symb{der}}{F})}{\avar})}}{(\app{(\app{\symb{der}}{G})}{\avar})}}
\]
This is not valid because in the right-hand side, both $F$ and $G$ occur with fewer arguments than
they have in the left-hand side.  A corrected version of this rule has the same left-hand side, but
the following right-hand side:
\[
\abs{\avar}{\real}{\app{\app{\symb{plus}}{
    (\app{(\app{\symb{der}}{(\abs{\bvar}{\nat}{(\app{F}{\bvar})})})}{\avar})
  }}{
    (\app{(\app{\symb{der}}{(\abs{\bvar}{\nat}{(\app{F}{\bvar})})})}{\avar})
  }}
\]
Note that in this rule, the left-hand side cannot be eta-shortened, so we could not avoid $F$ having
arity $1$.
\end{example}

Due to the pattern restriction, the type of the free variables in a rule can always be deduced from
context.  Hence, it suffices to give a rule without explicitly supplying $\Gamma$.

\subsection{Rewriting}

For a rule $\ell \arrz r$ with type environment $\Gamma$ and arity function $\arity$ from $\Gamma$
to $\N$, a \emph{meta-substitution} to a target type environment $\Delta$ is defined as a function
$\delta$ that maps each $\avar$ with $(\avar \hastype \atype) \in \Gamma$ and $\arity(\avar) = n$ to
a construction $\metaabsn{\bvar}{\btype}{n}{t}$, where
$\{\bvar_1,\dots,\bvar_n\} \in \V \setminus (\domain(\Gamma) \cup \domain(\Delta))$,
all $\bvar_i$ are distinct, and $\Delta \cup \{\bvar_1 \hastype \btype_1,\dots,\bvar_n
\hastype\btype_n\} \vdash t \hastype \atype$.

A meta-substitution $\delta$ is applied to $s \in \{\ell,r\}$, notation $s\delta$, by the following
procedure:
\begin{enumerate}
\item rename all bound variables that occur in $s$ to fresh variables, i.e., distinct variables
  that do not occur in $\Gamma$, $\Delta$, or any of the argument lists $\metaabslist{\bvar}{
  \btype}{n}$ in the range of $\delta$
\item replace each occurrence of $\apps{\avar}{s_1}{s_n}$ in $s$ with $\avar \in \FV(s)$ and
  $n = \arity(\avar)$ by $t[\bvar_1:=s_1\delta,\dots,\bvar_n:=s_n\delta]$ if
  $\delta(\avar) = \metaabsn{\bvar}{\btype}{n}{t}$; formally, this is achieved by:
  \begin{itemize}
  \item $(\apps{\afun}{s_1}{s_m})\delta = \apps{\afun}{(s_1\delta)}{(s_m\delta)}$
  \item $(\apps{\avar}{s_1}{s_m})\delta = \apps{\avar}{(s_1\delta)}{(s_m\delta)}$
    if $\avar \notin \domain(\delta)$
  \item $(\apps{\avar}{s_1}{s_n} \cdots s_m)\delta =
    \apps{(t[\bvar_1:=s_1\delta,\dots,\bvar_n:=s_n\delta])}{(s_{n+1}\delta)}{(s_m\delta)}$
    if $\delta(\avar) = \metaabsn{\bvar}{\btype}{n}{t}$ (by the arity restriction, we always
    have $0 \leq n \leq m$)
  \item $(\apps{(\abs{\avar}{\atype}{t})}{s_1}{s_n})\delta =
    \apps{(\abs{\avar}{\atype}{(t\delta)})}{(s_1\delta)}{(s_n\delta)}$; \\
    here, due to the $\alpha$-renaming beforehand, $x$ does not occur in $\domain(\delta)$ nor
    anywhere in any $\delta(\bvar)$
  \end{itemize}
\end{enumerate}

For a set of rules $\Rules$, the reduction relation on terms is defined by the following inductive
clauses:
\begin{itemize}
\item $\ell\delta \arr{\Rules} r\delta$ if $\ell \arrz r \in \Rules$ is a rule with type
  environment $\Gamma$ and arity function $\arity$, and $\delta$ is a meta-substitution on domain
  $\Gamma$ that respects $\arity$
\item $\app{(\abs{\avar}{\atype}{s})}{t} \arr{\Rules} s[\avar:=t]$ (the $\beta$-rule)
\item $\app{s}{t} \arr{\Rules} \app{s'}{t}$ if $s \arr{\Rules} s'$
\item $\app{s}{t'} \arr{\Rules} \app{s}{t'}$ if $t \arr{\Rules} t'$
\item $\abs{\avar}{\atype}{s} \arr{\Rules} \abs{\avar}{\atype}{s'}$ if $s \arr{\Rules} s'$
\end{itemize}

Note in particular that $\beta$ is a separate rule; we do not work \emph{modulo} $\beta$ (despite
the use of meta-substitutions for matching, which creates a somewhat similar effect).

\begin{example}\label{ex:map}
Suppose $\Rules$ contains a rule
\[
\app{\app{\symb{map}}{F}}{(\app{\app{\cons}{x}}{y})} \arrz
\app{\app{\cons}{(\app{F}{x})}}{(\app{\app{\symb{map}}{F}}{y})}
\]
We have that $\app{\app{\symb{map}}{\textcolor{red}{(\app{\symb{plus}}{(\app{\suc}{\nul})})}}}{(\app{\app{\cons}{\textcolor{blue}{\nul}}}{\textcolor{green}{\nil}})}$
reduces at the root to
\[\app{\app{\cons}{(\app{\textcolor{red}{\app{\symb{plus}}{(\app{\suc}{\nul})}}}{\textcolor{blue}{\nul}})}}{(\app{\app{\symb{map}}{\textcolor{red}{(\app{\symb{plus}}{(\app{\suc}{\nul})})}}}{\textcolor{green}{\nil}})}\]
using the meta-substitution $[F:=\metaabs{}{\textcolor{red}{\app{\symb{plus}}{(\app{\suc}{\nul})}}},x:=\metaabs{}{\textcolor{blue}{\nul}},t:=\metaabs{}{\textcolor{green}{\nil}]}$.

Also, $\app{
        \app{
          \symb{map}
        }{
          \textcolor{red}{(\abs{\avar}{\nat}{
            \app{\app{
              \symb{plus}
            }{
              (\app{\suc}{\nul})
            }}{
              \avar
            }
          })}
        }
      }{
        (\app{\app{\cons}{\textcolor{blue}{\nul}}}{\textcolor{green}{\nil}})
      }$
reduces at the root to
\[
  \app{
    \app{
      \cons
    }{(
      \app{
        \textcolor{red}{
          (\abs{\avar}{\nat}{
            \app{\app{
              \symb{plus}
            }{
              (\app{\suc}{\nul})
            }}{
              \avar
            }
          })
        }
      }{
        \textcolor{blue}{\nul}
      }
    )}
  }{
    (\app{
      \app{
        \symb{map}
      }{
        \textcolor{red}{
          (\abs{\avar}{\nat}{
            \app{\app{
              \symb{plus}
            }{
              (\app{\suc}{\nul})
            }}{
              \avar
            }
          })
        }
      }
    }{
      \textcolor{green}{\nil}
    })
  }
\]
using the meta-substitution $[F:=\metaabs{}{\textcolor{red}{\abs{\avar}{\nat}{\app{\app{\symb{plus}}{(\app{\suc}{\nul})}}{\avar}}}},
  x:=\metaabs{}{\textcolor{blue}{\nul}},t:=\metaabs{}{\textcolor{green}{\nil}]}$.
This in turn reduces by a $\arr{\beta}$ step to:
\[
  \app{
    \app{
      \cons
    }{(
      \app{
        \textcolor{red}{
            \app{
              \symb{plus}
            }{
              (\app{\suc}{\nul})
            }
        }
      }{
        \textcolor{blue}{\nul}
      }
    )}
  }{
    (\app{
      \app{
        \symb{map}
      }{
        \textcolor{red}{
          (\abs{\avar}{\nat}{
            \app{\app{
              \symb{plus}
            }{
              (\app{\suc}{\nul})
            }}{
              \avar
            }
          })
        }
      }
    }{
      \textcolor{green}{\nil}
    })
  }
\]
\end{example}

\begin{example}
In the above example, all variables have arity $0$.  Now suppose $\Rules$ instead contains a rule
where $F$ has arity $1$:
\[
\app{\app{\symb{map}}{(\abs{\cvar}{\nat}{(\app{F}{\cvar})})}}{(\app{\app{\cons}{x}}{y})} \arrz
\app{\app{\cons}{(\app{F}{x})}}{(\app{\app{\symb{map}}{(\abs{\cvar}{\nat}{(\app{F}{\cvar})})}}{y})}
\]
In this case, $\app{\app{\symb{map}}{\textcolor{red}{(\app{\symb{plus}}{(\app{\suc}{\nul})})}}}{(\app{\app{\cons}{\textcolor{blue}{\nul}}}{\textcolor{green}{\nil}})}$
\emph{does not reduce}, since every instance of the left-hand side has a $\lambda$-abstraction as the first argument of $\map$.

On the other hand,
$\app{
  \app{
    \symb{map}
  }{
    (\abs{\avar}{\nat}{
      \app{\textcolor{red}{\app{
        \symb{plus}
      }{
        (\app{\suc}{\nul})
      }}}{
        \avar
      }
    })
  }
}{
  (\app{\app{\cons}{\textcolor{blue}{\nul}}}{\textcolor{green}{\nil}})
}$
does reduce, to
\[
  \app{
    \app{
      \cons
    }{(
      \app{
        \textcolor{red}{
            \app{
              \symb{plus}
            }{
              (\app{\suc}{\nul})
            }
        }
      }{
        \textcolor{blue}{\nul}
      }
    )}
  }{
    (\app{
      \app{
        \symb{map}
      }{
        \textcolor{red}{
          (\abs{\avar}{\nat}{
            \app{\app{
              \symb{plus}
            }{
              (\app{\suc}{\nul})
            }}{
              \avar
            }
          })
        }
      }
    }{
      \textcolor{green}{\nil}
    })
  }
\]
(in one go), by using the meta-substitution
$[F:=\metaabs{y\!:\!\nat}{\app{\textcolor{red}{\app{\symb{plus}}{(\app{\suc}{\nul})}}}{y}},
  x:=\metaabs{}{\textcolor{blue}{\nul}},t:=\metaabs{}{\textcolor{green}{\nil}}]$.
Similarly, 
$\app{
  \app{
    \symb{map}
  }{
    (\abs{\avar}{\nat}{
      \textcolor{red}{\app{
        \app{\symb{plus}}{\textcolor{black}{\avar}}}{
        (\app{\suc}{\nul})
      }}
    })
  }
}{
  (\app{\app{\cons}{\textcolor{blue}{\nul}}}{\textcolor{green}{\nil}})
}$
reduces at the root to
\[
  \app{
    \app{
      \cons
    }{(
      \textcolor{red}{
        \app{
          \app{\symb{plus}}{\textcolor{blue}{\nul}}
        }{(\app{\suc}{\nul})}
        }
    )}
  }{
    (\app{
      \app{
        \symb{map}
      }{
        (\abs{\avar}{\nat}{
          \textcolor{red}{
            \app{\app{
              \symb{plus}
            }{
              \textcolor{black}{\avar}
            }}{
              (\app{\suc}{\nul})
            }
          }
        })
      }
    }{
      \textcolor{green}{\nil}
    })
  }
\]
\end{example}

\section{Potential subcategories}

The formalism described in this paper does not have a clear name in the literature, although it is
a slight variation (to be precise: a restricted version) of systems that appear in various sources
(e.g., \cite{bla:00,kop:12}).
Essentially, it is a simply-typed version of \emph{combinatory reduction systems}~\cite{klo:80},
with a $\beta$-rule implicitly included for all types.
We have imposed sanity limitations that often occur in the literature like limiting left-hand sides
to patterns and requiring right-hand sides to be in $\beta$-normal form; in the rare cases where
benchmarks are desired that violate these restrictions this can easily be encoded by including an
explicit ``application'' symbol $\symb{@}$ of the required type, and adding a rule
$\app{\app{\symb{@}}{(\abs{\avar}{\atype}{\app{F}{\avar}})}}{\bvar} \arrz \app{F}{\bvar}$.

There are other formalisms of higher-order term rewriting in the literature, several of which can be
viewed as limitations of, or strategies for, the formalism described here.  Hence, the same
benchmarks (or: a subset thereof) can be used in different categories should the need arise.  We
list the most pertinent of them below.

\subsection{Simply-typed Term Rewriting Systems~\cite{kus:01}}

In this case, the set of terms is restricted to \emph{$\lambda$-free terms}; that is, abstraction
is not permitted in term formation.  This limits both rule formation (no lambdas may occur on
either side of a rule), but also the terms that should be considered for termination or confluence.
Since terms contain no abstractions, the $\arr{\beta}$ step is not part of reduction.

\begin{example}
Let $\symb{A} \hastype \mathsf{o} \arrtype \mathsf{o} \arrtype{o}$ and
$\symb{L} \hastype (\mathsf{o} \arrtype \mathsf{o}) \arrtype{o}$, and
$\Rules = \{ \app{\symb{A}}{(\app{\symb{L}}{\avar})} \arrz x \}$.
(Note that this rule has type $\mathsf{o} \arrtype \mathsf{o}$.)
This system is:
\begin{itemize}
\item \emph{terminating} when viewed as an STRS, because every step decreases the size o fthe term
\item \emph{non-terminating} when viewed in the default way, because, writing
  $\omega = \app{\symb{L}}{(\abs{\avar}{\mathtt{o}}{\app{\app{\symb{A}}{\avar}}{\avar}})}$,
  we have $
  \app{\app{\symb{A}}{\omega}}{\omega} =
  \app{\app{\symb{A}}{(\app{\symb{L}}{\underline{(\abs{\avar}{\mathtt{o}}{\app{\app{\symb{A}}{\avar}}{\avar}})}})}}{\omega} \arr{\Rules}
  \app{(\abs{\avar}{\mathtt{o}}{\app{\app{\symb{A}}{\avar}}{\avar}})}{\omega} \arr{\beta}
  \app{\app{\symb{A}}{\omega}}{\omega}$.
\end{itemize}
\end{example}

\subsection{Plain higher-order rewriting~\cite{jou:rub:99}}

In this case, the set of terms is not restricted, but the set of \emph{rules} is: we require that
$\arity(\avar) = 0$ for all variables.  That is, variables should always occur unapplied in the
left-hand side, and meta-substitution is just substitution.  This means matching is \emph{plain}.
This is the formalism that was used in the higher-order category of the termination competition up
to 2024.

\begin{example}
In plain systems, there is no way to specify a rule
\[
\begin{array}{c}
\app{\symb{der}}{(\abs{\avar}{\real}{(\app{\app{\symb{plus}}{(\app{F}{\avar})}}{(\app{G}{\avar})})})}\ \arrz \\
\abs{\avar}{\real}{\app{\app{\symb{plus}}{
    (\app{(\app{\symb{der}}{(\abs{\bvar}{\nat}{(\app{F}{\bvar})})})}{\avar})
  }}{
    (\app{(\app{\symb{der}}{(\abs{\bvar}{\nat}{(\app{F}{\bvar})})})}{\avar})
  }}
\end{array}
\]
However, for systems that can be specified with arity $0$ for all variables (such as
Example~\ref{ex:map} and all STRSs), termination and confluence are equivalent whether we consider
the systems as plain or default (since the set of terms and the reduction relation are unchanged).
\end{example}

\subsection{Pattern higher-order rewriting systems~\cite{nip:91}}

This formalism as defined considers rewriting modulo $\alpha\beta\eta$.  However, it can be
modeled in our default formalism by (1) limiting the set of terms to terms in $\eta$-long form, and
(2) imposing a \emph{priority}: $\arr{\beta}$ must always be applied before any step using a rule.

\begin{example}
Let $\symb{h} \hastype (\mathsf{o} \arrtype \mathsf{o}) \arrtype \mathsf{o}$ and $\symb{a} \hastype \mathsf{o}$,
and $\Rules = \{ \app{\symb{h}}{F} \arrz \app{F}{(\app{\symb{h}}{(\abs{\avar}{\mathsf{o}}{\symb{a}})})} \}$.
This system is:
\begin{itemize}
\item \emph{terminating} when viewed as a PRS, since the priority ensures that, when
  $\app{\symb{h}}{(\abs{\avar}{\mathsf{o}}{\symb{a}})}$ is reduced, the result is immediately
  $\beta$-normalised to $\symb{a}$; termination of the whole system then follows by a computability
  argument;
\item \emph{non-terminating} when viewed in the default way, since
  $\app{\symb{h}}{(\abs{\avar}{\mathsf{o}}{\symb{a}})} \arr{\Rules}
  \app{(\abs{\avar}{\mathsf{o}}{\symb{a}})}{\underline{(\app{\symb{h}}{(\abs{\avar}{\mathsf{o}}{\symb{a}})})}}$,
  which contains the original term as a subterm.
\end{itemize}
\end{example}

Note that, for rules to be usable on terms in $\eta$-long form, they should match on $\eta$-long
terms, and result in them as well; hence, benchmarks can only be used as PRS benchmarks if, for all
rules $\ell \arrz r \in \Rules$, both $\ell$ and $r$ are well-behaved for the set $\FV(\ell)$:
\begin{itemize}
\item all variables in $\FV(\ell)$ are well-behaved;
\item $\abs{\avar}{\atype}{s}$ is well-behaved if $s$ is;
\item a term whose type is a sort is well-behaved if it has one of the forms
  $\app{\avar}{s_1} \cdots s_n$ or $\app{\afun}{s_1} \cdots s_n$, and all $s_i$ are well-behaved.
\end{itemize}
(That is, $\ell$ and $r$ are $\beta$-normal terms which are \emph{mostly} in $\eta$-long form,
except the free variables do not need to be expanded.)

\bibliographystyle{plain}
\bibliography{stcrs}

\end{document}

